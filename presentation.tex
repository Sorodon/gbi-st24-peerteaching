\documentclass{beamer}

\usepackage[ngerman]{babel}
\usepackage{array}
\usepackage{listings}
\usepackage{caption}
\usepackage{dirtytalk}
\usepackage{graphicx}

\beamertemplatenavigationsymbolsempty

\definecolor{codegreen}{rgb}{0,0.6,0}
\definecolor{codegray}{rgb}{0.5,0.5,0.5}
\definecolor{codepurple}{HTML}{C42043}
\definecolor{backcolour}{HTML}{F2F2F2}
\definecolor{bookColor}{cmyk}{0,0,0,0.90}  
\color{bookColor}

\lstset{upquote=true}

\lstdefinestyle{mystyle}{
    backgroundcolor=\color{backcolour},   
    commentstyle=\color{codegreen},
    keywordstyle=\color{codepurple},
    numberstyle=\footnotesize\color{codegray},
    stringstyle=\color{codepurple},
    basicstyle=\footnotesize,
    breakatwhitespace=false,         
    breaklines=true,                 
    captionpos=b,                    
    keepspaces=true,                 
    numbers=left,                    
    numbersep=-10pt,
    showspaces=false,                
    showstringspaces=false,
    showtabs=false,      
}
\lstset{style=mystyle} 

\title{Population Genetics}
\subtitle{Peerteaching KW24}
\author{Samuel Hehn \\ Swastik Kashyap}
\institute{Universität Tübingen}
\usetheme{Goettingen}
\date{\today}

\begin{document}

    {
    \setbeamertemplate{sidebar right}{}
    \begin{frame}
        \titlepage
    \end{frame}
    }

    {
    \setbeamertemplate{sidebar right}{}
    \begin{frame}
        \frametitle{Inhalt}
        \tableofcontents
    \end{frame}
    }

    \section{The coalescent process}

        \begin{frame}
            \frametitle{The coalescent process}
            \say{\textit{to coalesce}}: grow together, to join, to fuse

            \begin{definition}[coalescent event]<2->
                % TODO
                \textit{
                    If traversing the sequence-transmission paths backward in time,
                    two sequence transmission paths intersect at some sequence,
                    the paths coalesce at that intersection point. \\
                    This is called a coalescent event.
                }
            \end{definition}
        
            \begin{block}{Basic idea}<3->
                \begin{itemize}
                    \item<4-> Start with present-day generation
                    \item<5-> Construct previous generations
                    \item<6-> By randomly choosing parents in the previous generation
                \end{itemize}
            \end{block}
        \end{frame}

        \begin{frame}
            \frametitle{The coalescent process}
            \framesubtitle{Example}
            \only<2>{\includegraphics[page=1, width=\textwidth]{figures/coalescent_process.pdf}}
            \only<3>{\includegraphics[page=2, width=\textwidth]{figures/coalescent_process.pdf}}
            \only<4>{\includegraphics[page=3, width=\textwidth]{figures/coalescent_process.pdf}}
            \only<5>{\includegraphics[page=4, width=\textwidth]{figures/coalescent_process.pdf}}
            \only<6>{\includegraphics[page=5, width=\textwidth]{figures/coalescent_process.pdf}}
            \only<7>{\includegraphics[page=6, width=\textwidth]{figures/coalescent_process.pdf}}
            \only<8>{\includegraphics[page=7, width=\textwidth]{figures/coalescent_process.pdf}}
            
        
        \end{frame}

    \section{The standard coalescent model}
        \begin{frame}
            \frametitle{The standard coalescent model}
            Construct a tree based on estimated coalescent events. \\

            Needs more content

        \end{frame}

        \begin{frame}
            \frametitle{Coalescence of 2 genes}
            Considering a haploid model with $n$ genes. \\
            For two present day genes $i$ and $j$, when did they coalesce? \\
            \onslide<2->{We want two know two things:}
            \begin{enumerate}
                \item<3-> When did the two genes coalesce? \\
                      \onslide<4->{$\rightarrow$ Who is their common ancestor}
                \item<5-> How long is the waiting time until the two genes coalesced? \\
                      \onslide<6->{$\rightarrow$ How many generations back is their common ancestor?}
            \end{enumerate}
        \end{frame}

        \begin{frame}
            \frametitle{Coalescence of 2 genes}
            \framesubtitle{When did the two genes coalesce? }
                \onslide<2->{
                    We select a random ancestor for each individual: \\
                }
                \onslide<3->{
                    Probability to select the right ancestor of $i$ is $1$, since there are no requirements. \\
                }
                \onslide<4->{
                    Probability to select the right ancestor of $j$ is
                }
                \onslide<5->{
                    $$\frac{1}{n}$$
                    since we need to "hit" the ancestor we've chosen for $i$.
                }
        \end{frame}

        \begin{frame}
            \frametitle{Coalescence of 2 genes}
            \framesubtitle{How long is the waiting time until the two genes coalesced?}
            \onslide<2->{
                What is the Probability that the common ancestor is in Generation $n$? \\
            }
            \onslide<3->{
                ($n-1$ failures following one success)
            }
            \onslide<4->{
                \[
                    \left( 1 - \frac{1}{n} \right)^{n-1} \onslide<5->{\cdot \left( \frac{1}{n} \right)}
                \]
            }
        \end{frame}

        \begin{frame}
            \frametitle{Coalescence of k genes}

        \end{frame}

        \begin{frame}
            \frametitle{Continous time coalescent model}

        \end{frame}

\end{document}
